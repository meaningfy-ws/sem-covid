\chapter{Conclusions}
\label{sec:conclusions}
	
	This document presented the architectural stance for the LAM modelling initiative. Through the development of this architecture, we bring clarity of the project to the main stakeholders, what their interests and drivers are and what issues and solutions are associated to those motivations. We explicitly describe the internal processes, events and roles, answering questions concerning who shall do what and when. At a more specific level, valuable especially to the technical staff, is the application architecture, which answers the questions about what application service and capability supports which process in the LAM lifecycle. 
	
	This document constitutes a way to move forward with the digital transformation of the LAM team given its current interinstitutional context, management goals and demands from third parties. We aim to guide transitions in the asset source representation from the current unstructured and semi-structured sources towards structured representation of the source data. 
	
    The data quality is addressed in the current architecture through the introduction of manual and automatic verification and validation steps operating at both the form and meaning levels.
    
    This architecture sets the reference points for establishing the necessary capabilities and technologies for LAM data maintenance, processing and dissemination. However, it is not addressing the entire digital transformation in order to prevent disruption in the current production system. Rather, it's a part of the application that is foreseen to evolve, specifically the part responsible for editing the asset content (VocBench3 is already operational). The rest can be addressed in a subsequent step as a natural follow-up. 
	
	This architecture organises the business processes aiming to optimise the workflow process reducing the bottlenecks and increasing the speed of the overall lifecycle process.  

	\section{Summary}

	We presented in Section \ref{sec:context} the context of the current work given by the EU decisions and directives towards the semantic web technologies, open data and digital re-use of public sector information, along with implementation of a single digital gateway. The description of the state of play sets the baseline technical assessment which is extended by a recommendation of a joint trend towards the semantic web technologies and service oriented architecture.
	
	The architecture proposed here consists of four layers: motivation in Section \ref{sec:motivation-architecture}, business in Section \ref{sec:business-architecture}, application in Section \ref{sec:application-architecture} and this section.
	
    \section{Final word}
    
    In this document, we propose the first step towards a modern enterprise-level application, that streamlines the process of asset publication lifecycle from both the LAM team and for external partners involved in the process. 
    
    The offered vision is a service oriented and semantically enriched system that operates in a cloud infrastructure and provides seamless experience to all involved parties in performing their duties and responsibilities in a  coordinated asset lifecycle process. Such a system can constitute a cornerstone for management, publication and dissemination of public sector reference data bringing the single digital gateway one step closer to reality.